\documentclass[12pt,a4paper]{article}
\usepackage[utf8]{inputenc}
\usepackage[T1]{fontenc}
\usepackage[brazilian]{babel}
\usepackage{indentfirst}

\title{Sistema de Gerenciamento de Clientes}
\author{Daniel}
\date{\today}

\begin{document}
\maketitle

\section{Sistema de Clientes}

O sistema de gerenciamento de clientes foi implementado utilizando Haskell, uma linguagem 
funcional pura. O módulo permite o cadastro, edição, listagem e exclusão de clientes. 
Cada cliente possui um identificador único de 4 dígitos (entre 1000 e 9999), que é 
gerado automaticamente pelo sistema de forma sequencial, garantindo que não haja duplicidade. 
Além do ID, são armazenados o nome e o telefone do cliente. O sistema oferece uma interface 
por linha de comando que permite ao usuário navegar entre as diferentes funcionalidades 
através de um menu intuitivo. Durante a edição de um cliente, o sistema exibe as 
informações atuais e permite a modificação seletiva dos campos (nome e/ou telefone), 
mantendo o ID inalterado para preservar a integridade dos dados.

\end{document}
